\documentclass[11pt]{article}
\usepackage{amssymb,amsmath}
\newcommand{\R}{\mathbb R}
\newcommand{\st}{\text{subject to }}
\newcommand{\ip}[2]{\langle #1, #2 \rangle}
\newcommand{\twomat}[4]{\begin{bmatrix} #1 & #2 \\ #3 & #4 \end{bmatrix}}
\usepackage{hyperref}
\hypersetup{
    colorlinks=true,
    linkcolor=blue,
    filecolor=magenta,      
    urlcolor=cyan,
}
 
\title{Convex and Nonsmooth Optimization\\
HW4: Mostly about Semidefinite Programming}
\author{Michael Overton}
\date{Spring 2020}
\begin{document}
\maketitle
\begin{enumerate}
\item (20 pts) BV Ex 5.13
\item (40 pts) Consider the primal SDP 
\begin{align*}
\min & \ip{C}{X}\\
\st & \ip{A_i}{X}=b_i, \quad i=1,2\\
     &  X \succeq 0
\end{align*}
with 
\[
     C = \twomat{ 0}{1}{1}{0}, A_1=\twomat{1}{0}{0}{0}, A_2=\twomat{0}{0}{0}{1}, b_1=1, b_2=0
\]
\begin{enumerate}
\item Does the Slater condition hold for this primal SDP, i.e., does there exist a strictly feasible $\tilde X$?
\item What is the optimal value of the primal SDP?  Is it attained, and if so, by what $X$?
\item Write down the dual SDP.
\item Does the Slater condition hold for the dual SDP, i.e., does there exist a strictly feasible dual variable $\tilde y$?
\item What is the optimal value of the dual SDP? Is it attained, and if so, by what dual variable $y$?
\item Does strong duality hold?
\item What can you say in general about strong duality if the Slater condition holds for at least one of
a primal-dual pair of SDPs?

\end{enumerate}
\item (40 pts) Exercise 5.39 on p.~285--286 in BV (see also p.~219--220). 
Assume that the matrix $W$ is componentwise nonnegative, so that $W_{ij}=W_{ji}$ can be interpreted as a 
nonnegative weight on the edge joining vertex $i$ to vertex $j$. As well as answering the questions in the exercise, also do the following.
First, here is some useful information.
\begin{itemize}
\item The easiest way to set up an SDP in CVX is to use \mbox{``cvx\_begin sdp"} instead of \mbox{``cvx\_begin"}. Then, all matrix inequalities
 before the next ``cvx\_end" will be interpreted as semidefinite inequalities. Be sure to declare any symmetric matrix variables as symmetric, like this:
 ``variable X(n,n) symmetric". See
 \href{http://cvxr.com/cvx/doc/sdp.html}{here} for more details.
 \item If you declare a variable as ``dual variable Z" and then put \mbox{``: Z"} after an equality or inequality
 constraint, you will have access to the computed optimal dual variable for that constraint. If you want more
 than one dual variable, use ``dual variables Z y" (no comma).
 \item In {\sc Matlab}, and hence also CVX, if X is a matrix, diag(X) is a vector, and if x is a vector, diag(x) is a matrix. Type ``help diag" for more.
 \item Because the rank of a matrix is a discontinuous function, ``rank(X)" is not a reliable way to find the approximate rank of a matrix, especially one that has been computed with CVX. Instead, compute the eigenvalues with ``eig" (assuming X is symmetric) and estimate the rank from the
 eigenvalues.
 \end{itemize}

\begin{enumerate}
\item Using  $W$ from \href{http://www.cs.nyu.edu/overton/conv_ns_opt/hw/hw4data1.mat}{data set 1} and
 \href{http://www.cs.nyu.edu/overton/conv_ns_opt/hw/hw4data2.mat}{data set 2}, solve the SDP given in BV (5.114) with CVX.
 
 \item Also, solve the SDP given in BV (5.115) by CVX and compare its optimal value with the one for (5.114). 
 For the smaller data set 1, compare the computed optimal dual variables
 from (5.115) with the computed optimal primal variables from (5.114) and vice versa.
 What are the approximate ranks of the computed optimal primal and dual matrices? Do the matrices satisfy
 approximate complementarity, e.g.\ is the matrix product $X*Z$ approximately zero?
 
 \item Here is another way to motivate the SDP relaxation (5.115). Instead of insisting that the variables $x_i$ in (5.113) have the
 values $\pm 1$, replace
 each scalar $x_i$ by a vector $v_i\in\R^n$ with $\|v_i\|_2=1$, and then write $V=[v_1,\ldots,v_n]$ and $X=V^TV$. 
 Does such an $X$ satisfy the constraints in (5.115)? This is the motivation used
 in \href{http://www-math.mit.edu/~goemans/PAPERS/maxcut-jacm.pdf}{Goemans and Williamson's celebrated 1994 paper} 
 and this leads to a simple randomized procedure for assigning the vertices to the two sets: see equations (1)--(3) on p.~1120 of
 their paper. Solving the SDP gives you $X\succeq 0$ and then you need $V$ such that $X=V^TV$. Is such a $V$ unique? If not, what
 is a convenient choice? Show that it gives you $V$ whose columns have norm one as required. Would this work if $X$ is exactly low rank,
 or low rank to machine precision, instead of only approximately low rank? Why or why not?
 
 Using this $V$, carry out the assignment algorithm on p.~1120 for the data sets 1 and 2 using $r$ with $r_i=1/\sqrt{n}$ 
 (instead of a random vector) and print the resulting partitioning of the
 vertices and the corresponding cut value. How does it compare to the optimal value of the SDP?

  
 \item Explicitly solve (5.113) for data set 1 only.
 This problem is NP-hard so you will have to write a brute force method to solve it: there is no way to do it efficiently, but it should
 run fast enough on the smaller data set 1. According to Goemans and Williamson,
 the optimal value in their SDP relaxation (which CVX computes in polynomial time up to a given accuracy)
 should be within a factor of $\approx 0.878$ of the optimal value of the max cut problem. Is it? If not, perhaps there are some issues of scaling
 or constants that need working through. 
 \end{enumerate}
 
 An additional note of interest, not part of the homework: 
 H{\aa{}}stad showed that there is no polynomial time algorithm to improve
 this max-cut guaranteed approximation factor from 0.878 to $16/17\approx 0.941$ (assuming P $\not =$ NP), and 
 Courant's Subhash Khot and his collaborators showed in 
 \href{https://www.cs.cmu.edu/~odonnell/papers/maxcut.pdf}{this 2005 paper} that if Subhash's ``unique games conjecture" is true, then SDP is optimal for max-cut: one cannot get a better approximation
guarantee than $\approx 0.878$ in polynomial time unless P=NP. 
This would mean that SDP is somehow a very fundamental notion. Amazing!
 
\end{enumerate}
\end{document}
